\chapter{Executive summary}

The purpose of this document is for Smoking Games to be qualified to respond to the scenarios we have discovered and developed plans for during the development of our project. These plans will ensure that any scenario is handled efficiently without endangering human life, business function, classified information or compromising potential evidence. Though, for this to be the case, it will require full cooperation of the entire staff, spearheaded by the leaders. Without sufficient backing from the leaders of Smoking Games these plans would potentially do more harm than good, which is why we emphasize this part of the implementation process.\\
We started out by mapping Smoking Games's business functions, procedures, existing policies and guidelines before we analyzed our findings and documented all threats and business functions related to Smoking Games's business processes. This continued with weighing and analyzing Smoking Games's business functions and each threats, that have impact on the business. These two analyses are correlated and prioritized in the ”Scenario Priority Document” which is used to decide which scenarios are to be developed further.\\
Then we performed a deep analysis of each of the top ten scenarios from the ”Scenario Priority Document”, which led to the development of the plans according to the subordinate plan classification, section \ref{sec:SubordPlanClass}.\\
Due to Smoking Games's organizational size, we set on developing the top ten scenarios from the ”Scenario Priority Document”. This would secure Smoking Games from the ten most important scenarios in addition to avoid overwhelming them with plans that need training and maintenance.\\
The plans are a result of careful consideration of Smoking Games's internal and external business functions and policies. This ended up with an external document which is left out of the BCP document, since it could cause conflicts during an incident.\\
We have finished developing plans for two of the ten different scenarios, due to the time limitations of the project. This means there are still eight scenarios left for which a plan has to be developed.\\
The plans we have developed so far are described in the ”subordinate plan classification”, section \ref{sec:SubordPlanClass}; ”Fire” and ”Denial of Service attack”. Within these plans, we have made incident and disaster response plans for ”Fire” and ”Denial of Service attack”.\\
The result of this document will help Smoking Games responding to incidents and disasters way more efficiently than before. This plan will also ensure that Smoking Games will withstand serious blows to their business functions and still be able to resume business functions in a relatively short time.\\
Of course all of this depends on proper management, initiative and implementation of the developed plans.\\