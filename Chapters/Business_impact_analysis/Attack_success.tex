\section{Attack Success Scenario Development}\label{sec:Successful_Attack}
This section describes possible scenarios that could happen with great probability. The next tables use as template the "Example of A Malicious Code Attack Scenario" \cite{whitman4}. In each one of the scenarios we describe the threats, possible vulnerabilities, assets in danger and the actions to put in place in such scenario.

%%%%%%%%%%%		Fire scenario		%%%%%%%%%%%%%
\begin{longtable}{| p{4cm} | p{8cm} |}

	\hline \multicolumn{2}{| c |}{\textbf{Fire}}\\\hline
	\endfirsthead
	
	\hline \multicolumn{2}{| c |}{\textbf{Fire} ...continued}\\\hline
	\endhead
	
	\multicolumn{2}{|r|}{\textit{continued on next page}}\\\hline
	\endfoot
	
	\endlastfoot
	
	Date of analysis: & 15.07.2013 \\\hline
	
	Attack name/description: &  Fire in the Smoking Games headquarters\\\hline
	
	Threat agents: & 
	\begin{itemize}
	\item Electrical failure in the electric system.
	\item Use of lighters inside.
	\item Propagation of fire from colliding buildings.
	\end{itemize}\\\hline
	
	Known or possible vulnerabilities: &
	\begin{itemize}
	\item Even though the building is newly acquired the quality of the electrical installation is not optimal.
	\item Big probabilities of fire due to the location of the HQ.
	\item Smoking employees could try to smoke inside the building provoking a fire.
	\end{itemize}\\\hline
	
	Possible indicators: &
	\begin{itemize}
	\item Electrical failures will indicate possible problems in the circuits.
	\item After a long time without rain or water in general the probability of fire will increase a lot.
	\item Early detection of security threats inside the office. Like people smoking in the bathrooms.
	\end{itemize}\\\hline
	
	Information assets at risk from this attack: &
	\begin{itemize}
	\item Infrastructure and servers.
	\item Physical and digital media.
	\item User equipment.
	\item Employees
	\end{itemize}\\\hline
	
	Other assets at risk from this attack: &
	\begin{itemize}
	\item Office spaces.
	\item Building structure.
	\item External services.
	\end{itemize}\\\hline

	Immediate actions indicated when this attack is under way: &
	\begin{itemize}
	\item Shutdown all necessary systems
	\item Evacuate all personal and equipment and set up an external business.
	\end{itemize}\\\hline
	
	Actions in case of a successful scenario: &
	\begin{itemize}
	\item Review the actions taken in place.
	\item Prepare the environment of the external business.
	\item Start building reparations.
	\item Gradually move back to the main facility.
	\item Shutdown the external facilities used during the recovery.
	\end{itemize}\\\hline
	
	Comments: & None at this time. \\\hline

\end{longtable}

%%%%%%%%%%%		DDoS scenario		%%%%%%%%%%%%%
\begin{longtable}{| p{4cm} | p{8cm} |}

	\hline \multicolumn{2}{| c |}{\textbf{Denial of Service Attack}}\\\hline
	\endfirsthead
	
	\hline \multicolumn{2}{| c |}{\textbf{Denial of Service Attack} ...continued}\\\hline
	\endhead
	
	\multicolumn{2}{|r|}{\textit{continued on next page}}\\\hline
	\endfoot
	
	\endlastfoot
	
	Date of analysis: & 30.12.2013 \\\hline
	
	Attack name/description: &  Denial of service attack against the web services of Smoking games\\\hline
	
	Threat agents: & 
	\begin{itemize}
	\item Competitors.
	\item Script kiddies.
	\item Organized hackers.
	\end{itemize}\\\hline
	
	Known or possible vulnerabilities: &
	\begin{itemize}
	\item Overwhelmed servers.
	\item Malfunctioning firewalls or routers.
	\item Easy attack to perform on this days.
	\end{itemize}\\\hline
	
	Possible indicators: &
	\begin{itemize}
	\item Slow web services.
	\item Complains of the users.
	\item Previous trials performed in advance to test the system.
	\item Possible announcements from security firms.
	\end{itemize}\\\hline
	
	Information assets at risk from this attack: &
	Infrastructure and servers.\\\hline
	
	Other assets at risk from this attack: &
	None likely\\\hline

	Immediate actions indicated when this attack is under way: &
	\begin{itemize}
	\item Shutdown all necessary systems
	\item Replacement or reconfiguration of the attacked systems.
	\end{itemize}\\\hline
	
	Actions in case of a successful scenario: &
	\begin{itemize}
	\item Review the actions taken in place.
	\item Review of the possible compromised services.
	\item Alert employees of possible attacks during the close days to the attack.
	\end{itemize}\\\hline
	
	Comments: & None at this time. \\\hline

\end{longtable}
